%%% Abstract for Mars

We have built a new simplified 2.5D climate Model to simulate Mars during low obliquity epochs.
The model is derived from LMD's 3D global climate model (GCM), taking all key physical processes
into account, including long and short wavelength radiations (direct and scattered), thermal inertia
in the ground, CO$_2$ condensation and sublimation on the ground. This model computes the influence
of local slopes on the insolation as well as CO$_2$ glaciers flow on slopes. It uses simplified 
calculations for atmospheric transport so that we could simulate long enough to obtain an equilibrium that 
do not depend on the initial state. As we expected, lowering the obliquity causes Mars' atmosphere to collapse,
condensing its CO$_2$ onto the ground to form glaciers.
The influence of slopes and the flow of CO$_2$ glaciers both affect the localisation of perennial CO$_2$ glaciers
with CO$_2$ condensing and flowing on steep facing-poles slopes, at relatively low latitudes, in agreement 
with today's observations of moraines on Mars' ground. \cite{Kres:11}
Furthermore, our simulations
show interesting seasonal changes in atmospheric composition that are amplified during low obliquity epochs.
The atmosphere near the ground is depleted during seasons of condensation and enriched during seasons of
sublimation, with CO$_2$ mixing ratio oscillating between $x$ and $y$ percent. Not only this phenomenon acts
as a feedback on sublimation and condensation themselves, but it creates large gradients in atmospheric composition
that may have an impact on dynamic processes.
