We have built a new 3D Global Climate Model (GCM) to
simulate Pluto as observed by New Horizons in 2015.
All key processes are parametrized on the basis of theoretical
equations, including
atmospheric dynamics and transport,  turbulence,
radiative transfer, molecular conduction, 
as well as phases changes for N$_2$, CH$_2$ and CO. 
Pluto's climate and ice cycles are found to be very sensitive to model
parameters and initial states. Nevertheless, a reference simulation is designed by running a
fast, reduced version of the GCM with simplified atmospheric transport for 40,000 Earth years to
initialize the surface ice distribution and sub-surface temperatures, from which a
28-Earth-year full GCM simulation is performed. Assuming a topographic depression in a
Sputnik-planum (SP)-like crater on the anti-Charon hemisphere, a realistic Pluto is
obtained, with most N$_2$ and CO ices accumulated in the crater, methane frost
covering both hemispheres except for the equatorial regions, and a surface pressure near
1.1~Pa in 2015 with an increase between 1988 and 2015,  as reported 
from stellar occultations.
Temperature profiles are in qualitative agreement with the observations. In particular,
a cold atmospheric layer is obtained in the lowest kilometers above Sputnik Planum, as
observed by New Horizons's REX experiment. It is shown to result from the combined effect of the topographic
depression and  N$_2$ daytime sublimation. In the reference simulation with 
surface N$_2$ ice exclusively present in Sputnik Planum, the global circulation is only forced by
radiative heating gradients and remains relatively weak. Surface winds are locally 
induced by topography slopes and by N$_2$ condensation and sublimation around Sputnik
Planum.  However, the circulation can be more intense depending on the 
exact distribution of surface N$_2$ frost. This is illustrated in
an alternative simulation with N$_2$ condensing in the South Polar regions and N$_2$
frost covering latitudes between 35$^\circ$N and 48$^\circ$N. A global
condensation flow is then created, inducing strong surface winds everywhere, a prograde
jet in the southern high latitudes, 
and an equatorial superrotation likely forced by barotropic instabilities in
the southern jet. Using realistic parameters, the GCM predict atmospheric concentrations
of CO and CH$_4$ in good
agreement with the observations. N$_2$ and CO do not condense in the
atmosphere, but CH$_4$ ice clouds can form during daytime at low altitude near the regions
covered by N$_2$ ice (assuming that nucleation is efficient enough). 
This global climate model can
be used to study many aspects of the Pluto environment. For instance, organic hazes are
included in the GCM and analysed in a companion paper (Bertrand and Forget, Icarus, this issue).

