%%%%%%%%%%%%%%%%%%%%%%%%%%%%%%%%%%%%%%%%%%%%%%%%%%%%%%%%%%%%%%%%%%%%
%%%%%%%%%%%%%%%%%%%%%%%%%%%%%%%%%%%%%%%%%%%%%%%%%%%%%%%%%%%%%%%%%%%%
\section{Introduction}
%%%%%%%%%%%%%%%%%%%%%%%%%%%%%%%%%%%%%%%%%%%%%%%%%%%%%%%%%%%%%%%%%%%%
%%%%%%%%%%%%%%%%%%%%%%%%%%%%%%%%%%%%%%%%%%%%%%%%%%%%%%%%%%%%%%%%%%%%
\label{sc:Intro}

Presently, Mars' obliquity is 25.2$^{\circ}$, ressembling the Earth's. However,
contrary to the terrestrial obliquity, Mars' is chaotic and may have oscillated around
values as low as 10$^{\circ}$ during the past 250 Million years \citep{Lask:04}.
Low obliquity epochs have had an important impact on Mars' climate system,
causing the atmospheric CO$_2$ to condense. While permanent solid deposits of CO$_2$
were formed, the atmosphere was depleted and became much thinner. This evolution is 
often described as an atmospheric collapse. Some features that can presently be found
on Mars are thought to be evidences for such low obliquity epochs. Kreslavsky and Head
\citep{Kres:11} found ridges at relatively low latitudes (~70$^{\circ}$N) that they related
to drop moraines that are left by episodes of advance and retreat of cold-based glaciers.
They showed that those features were significantly different from landforms resulting from
water-ice flow and rather interpreted these as the consequence of CO$_2$ ice flow
during recent periods of very low obliquity.

In this paper, we quantitatively address the effect of low obliquity epochs on Mars'
environment. We study the resulting depletion of atmospheric CO$_2$ in term of drop of
pressure and seasonal local changes in atmospheric composition that are at stake in such
conditions. We focus on CO$_2$ ice deposition in term of total mass, localization and flow
in order to discuss the link between those conditions and geological evidence.

%%%%%%%%%%%%%%%%%%%%%%%%%%%%%%%%%%%%%%%%%%%%%%%%%%%%%%%%%%%%%%%%%%%%
%%%%%%%%%%%%%%%%%%%%%%%%%%%%%%%%%%%%%%%%%%%%%%%%%%%%%%%%%%%%%%%%%%%%
\section{Model description}
%%%%%%%%%%%%%%%%%%%%%%%%%%%%%%%%%%%%%%%%%%%%%%%%%%%%%%%%%%%%%%%%%%%%
%%%%%%%%%%%%%%%%%%%%%%%%%%%%%%%%%%%%%%%%%%%%%%%%%%%%%%%%%%%%%%%%%%%%

\label{sc:model}
\label{sc:physi}


%%%%%%%%%%%%%%%%%%%%%%%%%%%%%%%%%%%%%%%%%%%%%%%%%%%%%%%%%%%%%%%%%%%%
\subsection{Generalities}
%%%%%%%%%%%%%%%%%%%%%%%%%%%%%%%%%%%%%%%%%%%%%%%%%%%%%%%%%%%%%%%%%%%%

\label{sc:dynam}
\label{sc:dynamic}

The model we used is derived from the LMD Mars GCM \cite[]{Forg:99}. This model is made of a
physical core computing physical processes and a dynamic core that performs temporal and 
spatial integration of the equations of hydrodynamics. We choosed to drop the dynamic core of
the initial model and replaced it with a simpler redistribution scheme detailled below in 
order to ensure quicker simulations.

In this paper, we present simulations with a horizontal grid of 32$\times$48, that is a 
grid-point spacing of 11.25$^{\circ}$ in longitude by 3.75$^{\circ}$ in latitude. This 
configuration allowed us to obtain accurate estimates of the latitudes where solid CO$_2$
can be deposited. In term of vertical resolution, the model uses the terrain-following 
"sigma" coordinate system in finite difference form (i.e. each layer is defined by a 
constant value of the ratio pressure devided by surface pressure). As we worked in 
conditions where the atmosphere is much thinner than presently, we choosed to divide the
vertical coordinate in only 3 layers, with the pressure for the upper layer of %To complete%
which corresponds to a spatial resolution of about %To complete% 
for the case of 10$^{\circ}$ obliquity.

We simulated the evolution of the planet during timescales ranging between 1000 and 20000
martian years. As we did not need to resolve diurnal climate variations, we choosed timesteps
of 10 martian days each for simulations. We found that these timesteps were sufficiently 
short to provide interresting insights on seasonal evolutions of the climate over a year.


%%%%%%%%%%%%%%%%%%%%%%%%%%%%%%%%%%%%%%%%%%%%%%%%%%%%%%%%%%%%%%%%%%%%
\subsection{Model for dynamics}
%%%%%%%%%%%%%%%%%%%%%%%%%%%%%%%%%%%%%%%%%%%%%%%%%%%%%%%%%%%%%%%%%%%%
\label{sc:model_dynamics}

We replaced the dynamic core of the original GCM by a newtonian return to the mean,
that was already  used by Bertrand et al. to simulate the climate on Pluto on large timescales
\citep{Forg:17}. We applied this redistribution scheme to potential temperature, surface
pressure and CO$_2$ mixing ratio. For any mesh $i$, in any vertical layer $l$ between times
$t$ and $t + \delta t$, we apply:

\begin{equation}
\label{rappel}
	\theta_{(i,l)}(t+\delta t) = \theta_{(i,l)}(t) + \left( \overline{\theta_{l}}(t) -  \theta_{(i,l)}(t) \right) \left( 1 - e^{\frac{\delta t}{\tau_\theta}} \right)
\end{equation}
\begin{equation}
	q_{(i,l)}(t+\delta t) = q_{(i,l)}(t) + \left( \overline{q_{l}}(t) -  q_i(t) \right) \left( 1 - e^{\frac{\delta t}{\tau_q}} \right)
\end{equation}
\begin{equation}
	Ps_i(t+\delta t) = Ps_i(t) + \left( P_0 k_i(t) -  Ps_i(t) \right) \left( 1 - e^{\frac{\delta t}{\tau_P}} \right)
\end{equation}

where $\theta_{(i,l)}$ and $q_{(i,l)}$ are respectively the potential temperature and the CO$_2$ 
mixing ratio of mesh $i$ in layer $l$, $Ps_i$ is the surface pressure of mesh $i$. The return is
made toward the mean values $\overline{\theta_{l}}$, $\overline{q_{l}}$ and $P_0 k_i$ where 
$k_i = e^{-\frac{z_i g}{RT}}$ with $z_i$ the elevation of the ground. Time constants for the 
return are $\tau_\theta$, $\tau_q$ and $\tau_P$. The constraints of total mass and energy
conservation lead to:

\begin{equation}
	\overline{\theta_{l}} = \frac{ <\theta_{.,l} \Delta P_{l}>}{< \Delta P_{l} >}
\end{equation}
\begin{equation}
	\overline{q_{l}} =  \frac{< \Delta P_{l} q >}{< \Delta P_{l} >}
\end{equation}
\begin{equation}
	P_0 = \frac{< Ps >}{< k >}
\end{equation}

where $\Delta P_{l}$ is the pressure loss in layer $l$ and $f \rightarrow < f >$ is the 
surface-average operator. We compared test simulations between our model and the original GCM 
to tune the value of the three time constants $\tau_\theta$, $\tau_q$ and $\tau_P$,
so that our model gives a good account of the evolution of the climate in average. In our
simulations, we use $\tau_\theta = 10^{-7}$s, $\tau_q = 10^{-5}$s and $\tau_P = 1$s


%%%%%%%%%%%%%%%%%%%%%%%%%%%%%%%%%%%%%%%%%%%%%%%%%%%%%%%%%%%%%%%%%%%%
\subsection{Insolation on slopes}
%%%%%%%%%%%%%%%%%%%%%%%%%%%%%%%%%%%%%%%%%%%%%%%%%%%%%%%%%%%%%%%%%%%%
\label{sc:insolation_slopes}

Kreslavsky and Head studied with a simple energy balanced model the atmospheric collapse and 
deposition of solid CO$_2$ in low obliquity conditions \citep{Kres:05}. They insisted in their
article on the role of topographic slopes on the  insolation regime. Regarding 
their results, we found important to take local slopes into account. Given the extension of the 
mesh grid (more than 100 km large), associating each grid area to a singe slope would 
not be relevant. To characterize the slopes on a mesh, we binned its local slopes extracted from
MOLA observations (with \textasciitilde 300 m resolution) into 7 characteristic slopes, based on
their value of $\mu = \theta ~cos(\psi)$ where $\theta$ and $\psi$ are the local inclination and 
orientation of the slope, meaning that $\mu$ is the projection of the orientation on the 
south-north axis. We found that slopes having close parameters $\mu$ also receive close mean daily
insolation.

Computation of the insolation on each of the characteristic slopes is done accordingly to Spiga
and Forget article on estimations of the solar irradiance on Martian slopes using 3D Monte-Carlo
calculations \citep{Spig:08grl}. Using this method, the mean daily insolation is estimated on the
first day of the timestep, assuming that it approximately remains the same during the 10-days timestep.
Similarly, surface temperature, soil temperature and mass of CO$_2$ ice deposits are computed separately
on each characteristic slope. 


%%%%%%%%%%%%%%%%%%%%%%%%%%%%%%%%%%%%%%%%%%%%%%%%%%%%%%%%%%%%%%%%%%%%
\subsection{CO$_2$ glaciers flow}
%%%%%%%%%%%%%%%%%%%%%%%%%%%%%%%%%%%%%%%%%%%%%%%%%%%%%%%%%%%%%%%%%%%%
\label{sc:glaciers_flow}

\citep{Fast:17} used models of rheology to compute maximum thickness and velocity of CO$_2$ ice in
the conditions of sloppy topography. They showed that, even on tiny slopes, the CO$_2$ ice sheet is
very thin and has high velocity: looking at the maximum inclination of 0.1$^{\circ}$ they studied,
we find that the related maximum thickness remains under 100m.
Among the 7 characteristic slopes we choosed to describe the topography, the 6 that are not flat have
inclinations ranging between 6$^{\circ}$ and 31.5$^{\circ}$, being much steeper than the ones studied
by \citep{Fast:17}. We thus expect the CO$_2$ ice layer upon those slopes to be very thin and of
high velocity, meaning they would not impact quantitatively the geographic deposition of CO$_2$
while they would errode and shape the ground on which they are moving fast. Also, we expect that the
steeper the slope, the more likely CO$_2$ deposition is to occur.

Those preliminary observations lead us to designing a very simple model to simulate the flow of CO$_2$. 
We arbitrary defined a maximum thickness of 10m on every characteristic slopes that are not flat. When
the CO$_2$ mantle reach this thickness on one of those characteristic slopes, all further deposition 
will be directly transmitted to the slope of neighboring inclination. This phenomenon account for the
flow of CO$_2$ ice from the slopiest ground to the flat ground. We found during that taking
into account the CO$_2$ flow has an important impact on simulations concerning the time needed 
to reach an equilibrium when changing the obliquity.




%%%%%%%%%%%%%%%%%%%%%%%%%%%%%%%%%%%%%%%%%%%%%%%%%%%%%%%%%%%%%%%%%%%%
%%%%%%%%%%%%%%%%%%%%%%%%%%%%%%%%%%%%%%%%%%%%%%%%%%%%%%%%%%%%%%%%%%%%
\section{Initial parameters and simulations}
%%%%%%%%%%%%%%%%%%%%%%%%%%%%%%%%%%%%%%%%%%%%%%%%%%%%%%%%%%%%%%%%%%%%
%%%%%%%%%%%%%%%%%%%%%%%%%%%%%%%%%%%%%%%%%%%%%%%%%%%%%%%%%%%%%%%%%%%%
\label{sc:init}


Although we have accurate models and observations of the present Mars' environment, estimating it
for very different obliquities remains a difficult task. To obtain rough estimates of what Mars looks
like with a certain obliquity, we carried out preliminary simulations with constant values of orbital
parameters. Those simulations provided us with initial states for further simulations in which we
took into account the variation of orbital parameters.

%%%%%%%%%%%%%%%%%%%%%%%%%%%%%%%%%%%%%%%%%%%%%%%%%%%%%%%%%%%%%%%%%%%%
\subsection{Preliminary simulations}
%%%%%%%%%%%%%%%%%%%%%%%%%%%%%%%%%%%%%%%%%%%%%%%%%%%%%%%%%%%%%%%%%%%%
\label{sc:preli_simu}

As a first step, we ran our model without excentricity and with constant obliquities ranging between 
0$^{\circ}$ and 15$^{\circ}$ for 10000 years. The initial state for those simulations corresponds
to present Mars, with obliquity 25.2$^{\circ}$ and excentricity 0.05. 


%%%%%%%%%%%%%%%%%%%%%%%%%%%%%%%%%%%%%%%%%%%%%%%%%%%%%%%%%%%%%%%%%%%%
\subsection{Reference simulations}
%%%%%%%%%%%%%%%%%%%%%%%%%%%%%%%%%%%%%%%%%%%%%%%%%%%%%%%%%%%%%%%%%%%%
\label{sc:ref_simu}
